%Learning Shallow Systems
This section investigates some properties of non-collapsing shallow theories that will be useful in the following sections.  
We introduce a representation for non-collapsing shallow theories known as \emph{maximally-generalized signature equations}.
We will show that this representation is canonical for ground-equivalent theories up to a renaming of variables.
This representation can be used to determine a canonical presentation for non-collapsing shallow theories. 
Assuming a fixed signature, we will also show that the size of this representation is polynomial in the size of any ground-equivalent presentation. 

Throughout this section, unless otherwise stated, we will assume that every theory is non-trivial. 
This means that there are at least two ground terms that are equivalent and at least two ground terms that are not equivalent. %\todo{odd sentence}. 
%The algorithm can query the presentations $\emptyset$ and $\{x \approx y\}$ to the oracle in order to rule out the trivial cases.  \todo{ do we want to assume this? We need to alter it to now talk about algorithms, as well.}

The proofs for the statements made in this section are given in an appendix at the end of the paper.  


\subsection{Signature Equations}
Given an equational theory $E$, a \emph{signature}, \emph{sig} (defined over $E$), consists of a function symbol $f \in \Sigma_k$ and a set of equivalence classes $C_1, \dots, C_k \in EQ_G(E)$, represented $\ang{f,C_1,\dots,C_2}$. %$[s_1]_E, \dots, [s_k]_E$. 
We call $f$ the \emph{head} and $C_1, \dots, C_k$ the \emph{body} of the signature.%\todo{we use head once below} of the signature (written $f = head(sig)$)
 We write $sig[i]$ for each $C_i$. %\todo{does this 'written' make sense?}

An \emph{instance} of \emph{sig} is a term $f(s_1,\dots,s_k)$, where for each $i$,  $s_i \in C_i$.
We write $\inst{sig}$ to represent the set of instances of $sig$.
We may also represent \emph{sig} by $\ang{f,s_1,\dots,s_k}$ when $f(s_1,\dots,s_k) \in \inst{sig}$, since each $s_i$ acts as a representative element of $C$. 
%\todo{do we need to define signatures, or should we just jump to extended signatures?}

An \emph{extended signature}, $sig'$ is defined analogously, but may contain variables in place of equivalence classes.
Moreover, instances of extended signatures may replace like variables with like terms.
When an instance contains all the same variables as its signature, it is a maximall-generalized instance
For example $\ang{f,a,x,x}$ and $\ang{f,[a]_E,x,x}$ are both representations of the same extended signature, and $f(a,x,x)$ and $f(a,b,b)$ are instances of this signature, though $f(a,x,x)$ is the only maximally-generalized instance. %\todo{do we want to let instances allow for substitutions as well?}
%Extended signatures may also contain a single variable, such as $x$, written $\ang{x}$.
Given two signatures, $sig_1$ and $sig_2$, we define the order $\preceq$ such that  $sig_1 \preceq sig_2$ if $\inst(sig_1) \subset \inst(sig_2)$.
We say $sig_1 \prec sig_2$ if $sig_1 \prec sig_2$ but $sig_2 \not\prec sig_1$.
%\todo{define representative instance vs instance (or some other names)}
For example, $\ang{f,a,b,b} \prec \ang{f,a,x,x}$.
Likewise, $\ang{f,a,x,b} \not\prec \ang{g,a,x,b}$ since they have different heads, and $\ang{f,a,x,y} \not\prec \ang{f,x,x,y}$. %\todo{say why}
If $sig_1 \prec sig_2$, we say that $sig_2$ is more general than $sig_1$.
For a set $I$ of indices, we write $sig[c]_I$ to replace each element at each $i \in I$ with $c$.

A \emph{signature equation} is an pair of extended signatures, written $sig_1 \approx sig_2$ for signatures $sig_1$ and $sig_2$.
An \emph{instance} of a signature equation $sig_1  \approx  sig_2$ is an equation $s_1 \approx s_2$, where $s_1$ is an instance of $sig_1$ and is an instance $s_2$ of $sig_2$.
The signature equation $sig_1 \approx sig_2$  (defined over $E$) \emph{holds} on $E$ if for every ground instance $s_1 \approx s_2$ of $sig_1 \approx sig_2$, $s_1 \approx_E s_2$.
A \emph{maximally generalized} signature equation (MGSE) is a signature equation $sig_1 \approx sig_2$ defined over $E$ such that for all signatures $sig'_1$ and $sig'_2$ such that $sig_1 \preceq sig'_1$ and $sig_2 \preceq sig'_2$, there is an instance $s'_1 \approx s'_2$ of the equation such that $s'_1 \ne_E s'_2$. 
The set of maximally generalized signature equations of a theory $E$ is written $MGSE(E)$.
Every pair of equivalent ground terms is an instance of some MGSE.
Note that for any presentation $E'$, it holds that $E \greq E'$ if and only if $MGSE(E) = MGSE(E')$ (up to a renaming of variables). %\todo{should I bother to write out this proof? Or state this in a proposition?}
%Therefore, we can try to learn an equational theory by learning its MGSE's



An \emph{essential class} of a presentation $E$ is an equivalence class that appears in the body of some signature in $MGSE(E)$. %\todo{should we consider all constants to be essential classes?}
For example, if $E := \{ f(g(a),x) \approx b\}$, then $[g(a)]$ and $[b]$ are essential classes, since $\ang{f,g(a),x} \approx \ang{b}$ are MGSEs.
Then the only MGSE of $E := \{ f(a) \approx b \}$ is $\ang{f,a} \approx \ang{b}$, so its only essential class is $[a]$, since $b$ is in the head, not the body, of the equation.
The set  $E := \{ a \approx b, f(x) \approx g(x) \}$ has MGSEs $\ang{f,x} \approx \ang{g,x}$ and $\ang{a} \approx \ang{b}$, and it has no essential classes.
Lastly, the set $E:= \{ f(x,y) \approx f(y,x), f(a,x)\approx b \}$ has MGSEs $\ang{f,x,y} \approx \ang{f,y,x}$, $\ang{f,a,y}\approx \ang{b}$, and $\ang{f,x,a} \approx \ang{b}$.

Let $\ec(E)$ denote the set of of essential classes in $E$.
We will use $\ec$, when $E$ is clear from context. 

The following lemma will be useful to identify essential terms.

\begin{lemma}
\label{ess_id}
Let $s \approx t \in Th_G(E)$ and let $u$ be a term with $I := D1P_{[u]}(s)$ and $J := D1P_{[u]}(t)$.
Then  $u$ is an essential term if and only if there is a $v$ such that $s[v]_I \approx t[v]_J \not\in Th_G(E)$.
\end{lemma}



\subsection{The Canonical Representation $E_{rep}$}
Let $\tpre$ be any total ordering on terms such that for any terms $s$ and $t$, $\|s\| < \|t\|$ implies $s \tpre t$.
For a fixed $\tpre$, we will see how to construct a presentation, $E_{rep}$, of $E$ that is canonical up to a renaming of variables.
Only variables and essential terms appear at depth 1 in any equation in $E_{rep}$.

For each $C \in \ec$, let the \emph{representative term} of $C$, called $rep_C$, be minimal ground term in $C$ with respect to $\tpre$ . %\todo{should every subterm also be representative?}
Take any $sig_1 \approx sig_2 \in MGSE(E)$, where $sig_1 := \ang{f,u_1,\dots,u_k}$ and $sig_2 := \ang{g,u_{k+1},\dots,u_{k+r}}$ and for each $i$, $u_i \in \ec \cup X$. 
We define the \emph{representative equation} of $sig_1 \approx sig_2 \in MGSE(E)$ to be the equation $f(u'_1,\dots,u'_k) \approx g(u'_{k+1},\dots,u'_{k+r})$, where $u'_i := u_i$ if $u_i \in X$ and $u'_i := rep_C$ if $u_i \in C \in \ec$. 
Let $E_{rep}$ be the set of representative equations of $MGSE(E)$.
The proof that $E_{rep} \greq E$ is given in the following proposition.

\begin{proposition}
\label{erep}
Given the presentation $E_{rep}$ constructed from  $MGSE(E)$ as above with ordering $\tpre$, $E_{rep} \greq E$.
\end{proposition}

%Note that the above proof also suggests a polynomial time algorithm to find a derivation from $s$ to $t$ in $E_{rep}$. \todo{make sure derivation is defined}


\subsection{Bounding the Size of $E_{rep}$}

We have seen how to construct a canonical presentation $E_{rep}$ from the essential classes $\ec$ of a presentation $E$.
 But what if $\|\ec\|$ is very large?
 This section gives a polynomial bound for $\|\ec\|$ in terms of the size of any ground-equivalent non-collapsing shallow presentation $E$.
 We use this to give polynomial bounds for $|E_{rep}|$ and $\|E_{rep}\|$ in terms of $|E|$ and $\|E\|$, respectively.

\begin{lemma}
\label{d_one_lemma}
Let $E$ be any non-collapsing shallow presentation and let $s, t$ be terms in $T(\Sigma, X)$ such that $s \approx_E t$.
Assume there is a $u \in T(\Sigma)$ such for every $s' \approx t' \in E$, $[u]\not\in D1(s') \cup D1(t')$. % and every $n \in \mathbb{N}$, $s' |_n \not \approx_E u$ and $t' |_n \not \approx_E u$. \todo{is the last statement redundant with t' ?} \todo{should we give a name to what $u$ is?} \todo{just say $u$ doesn't appear at depth 1 in E}
Let $I := D1P_{[u]}(s)$ and $J := D1P_{[u]}(t)$.
Then $s[x]_I \approx_E t[x]_J$ for some $x \in X \backslash (Vars(s) \cup Vars(t))$. 
\end{lemma}


\begin{lemma}
\label{eq_d1}
For any non-collapsing shallow presentation $E$ such that $|EQ_G(E)| \ge |D1(E)| + 2d$, $\ec(E) \subseteq D1(E)$.
\end{lemma}


The condition that $|EQ_G(E)| \ge |D1(E)| + 2d$ is in fact necessary for the above lemma, though the inequality isn't necessarily tight.
Specifically, if there is a non-collapsing shallow $E$ such that $|D1(E)| + 2d \ge |EQ_G(E)|$, then it is not necessary that $\ec(E) \subseteq D1(E)$.
For example, let $\Sigma := \{ f:2, a:0, b:0 \}$ and let $E := \{ f(x,a)\approx a, f(x,x)\approx a, f(x,f(a,b))\approx a, f(f(a,b),x)\approx a\}$, so $D1(E) = \{ [a], [f(a,b)]\}$.
A quick check will show that $EQ_G(E) = \{ [a], [b], [f(a,b)]\}$. 
But this leaves $\ang{f,b,x}\approx \ang{a}$ as an MGSE, though $[b] \not\in D1(E)$.

This lemma yields a simple but important corollary.

\begin{corollary}
For any non-collapsing shallow presentation $E$, $|\ec(E)| \le 2d + |D1(E)|$.
\end{corollary}

We can also use the above lemma to prove a bound on how much larger $E_{rep}$ is than any other presentation of the same theory.

\begin{theorem}
\label{eqs_bound}
Let $E$ be any non-collapsing shallow presentation over the alphabet $\Sigma$, and let $E_{rep}$ be the representative presentation formed from the essential classes of $E$. 
Then $|E_{rep}| \le |\Sigma|^2(2d |E| + 4d)^{2d}$. 
\end{theorem}

Note that it is not possible to bound $|E_{rep}|$ in terms of $\|E\|$.
For example, for any positive integer $k$, let $E := \{ f^k(a) \approx a \}$, where $f^k(a)$ is the result of applying $f$ to $a$ $k$ times.
Then $|E| = |E_{rep}| = 1$, but $\|E\| \ge k$ for any $k$.

We can also prove a polynomial bound on $\|E_{rep}\|$ depending on $\|E\|$ and $|E|$ for any $E \greq E_{rep}$.
In the following, given a complexity class $c$ of $E$ let $minsize_E(c)$ represent the value $\|t\|$ that is minimized for all $t \in c$.


\begin{theorem}
\label{size_bound}
Let $E$ be any non-collapsing shallow presentation over the alphabet $\Sigma$, and let $E_{rep}$ be the representative presentation formed from the essential classes of $E$. 
Then $\|E_{rep}\| \le |E_{rep}|(\|E\|(2d)^{2d}+2)$. 
\end{theorem}
