Equational logic is a formalism used to describe infinite sets of equations between terms (\emph{theories}) using finite sets of equations (\emph{presentations}). 
Both functional programs and logic programs can be naturally represented as equational logic presentations. 
Therefore, learning presentations in equational logic can be seen as a form of program synthesis. 
In general, it is undecidable whether terms are equivalent via a finite presentation.
So learning equational presentations is generally intractable. 

This paper investigates the exact learning of a restricted class of theories known as \emph{non-collapsing shallow theories}, which can be presented by equations where variables only appear at depth one. 
The learning algorithms use examples and queries of equations between ground terms, meaning there are no variables in the equations.
It is shown that these theories cannot be learned in the limit from only positive examples. 
A polynomial time algorithm is given which creates a hypothesis presentation consistent with positive and negative examples which will learn in the limit a presentation for the target theory.
Finally, an algorithm is given which learns a presentation in polynomial time from a minimally adequate teacher.
It is shown that the learned presentations are canonical with respect to an ordering on terms and the presentation is at most polynomially larger than the minimal presentation of the same theory.